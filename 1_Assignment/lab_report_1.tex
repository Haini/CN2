%%%%%%%%%%%%%%%%%%%%%%%%%%%%%%%%%%%%%%%%%
% CN2 Labreport template
%
% License:
% CC BY-NC-SA 3.0 (http://creativecommons.org/licenses/by-nc-sa/3.0/)
%
%%%%%%%%%%%%%%%%%%%%%%%%%%%%%%%%%%%%%%%%%

\documentclass[parskip=full]{scrartcl}

\usepackage{siunitx}  % Provides the \SI{}{} command for typesetting SI units
\usepackage{graphicx} % Required for the inclusion of images
\usepackage{booktabs} % nicer tables
\usepackage{hyperref}
\usepackage[noabbrev]{cleveref} % automatic references
\usepackage{listings} % typeset code

\crefname{lstlisting}{listing}{listings} % for referencing code
\Crefname{lstlisting}{Listing}{Listings} % for referencing code

\usepackage[headsepline]{scrlayer-scrpage} % header
\ohead{Group 08} % right part of header
\ihead{Assignment 1} % left part of header

\lstset{basicstyle=\ttfamily} % monospaced font in listing

\usepackage{lstautogobble}  % Fix relative indenting
\usepackage{color}          % Code coloring
\usepackage{zi4}            % Nice font

\definecolor{bluekeywords}{rgb}{0.13, 0.13, 1}
\definecolor{greencomments}{rgb}{0, 0.5, 0}
\definecolor{redstrings}{rgb}{0.9, 0, 0}
\definecolor{graynumbers}{rgb}{0.5, 0.5, 0.5}

\usepackage{listings}
\lstset{
    autogobble,
    columns=fullflexible,
    showspaces=false,
    showtabs=false,
    breaklines=true,
    showstringspaces=false,
    breakatwhitespace=true,
    escapeinside={(*@}{@*)},
    commentstyle=\color{greencomments},
    keywordstyle=\color{bluekeywords},
    stringstyle=\color{redstrings},
    numberstyle=\color{graynumbers},
    basicstyle=\ttfamily\footnotesize,
    frame=l,
    framesep=12pt,
    xleftmargin=12pt,
    tabsize=4,
    captionpos=b
}



%----------------------------------------------------------------------------------------
%	DOCUMENT INFORMATION
%----------------------------------------------------------------------------------------

\begin{document}
\begin{titlepage}
    \centering
    \vspace*{2cm}
    {\Huge \textbf{Communication Networks 2}}\\
    SS 2019\\
    \vspace*{1cm}
    {\Large Assignment 1}
    \\\vspace*{3cm}
    {\Large \textbf{Group 08}}\\
    \vspace*{1cm}
    {\large 
        \begin{tabular}{l c c}
            Name & Mat. Number \\ \hline
            Constantin SCHIEBER & 01228774 \\
            Andreas HIRTENLEHNER & 01327273
        \end{tabular}
    }
    \\\vspace*{7cm}
    \today
\end{titlepage}

%----------------------------------------------------------------------------------------
%	SECTION 1
%----------------------------------------------------------------------------------------
\section{Email Password Recovery}

\subsection{Description of the solution}

\subsubsection{Used Protocols}
Thunderbird uses the protocol IMAP4rev1 (see RFC3501 6.2.2) for the communication with the email server.
The framework SASL (Simple Authentication and Security Layer, see RFC442 / RFC 2222) is used for authentification purposes.
In the labrotory set-up SASL was configured with the PLAIN flag, which means that all data is exchanged in plain text.

\subsubsection{Observed Thunderbird Traffic}
We can observe that Thunderbird negotiates its supported capabilities in Listings \ref{lst:protocol} and \ref{lst:protcol2}.
Listing \ref{lst:protocol} shows us that a plain-text authentication process is used, the password should therefore be available in plain-text in one of the following packets.

\begin{lstlisting}[language=tex, breaklines, frame=single, caption={Used Protocol IMAPv4rev1, SASL-IR for authentication}, label=lst:protocol, float, floatplacement=h]
* OK [CAPABILITY IMAP4rev1 LITERAL+ SASL-IR LOGIN-REFERRALS ID ENABLE IDLE AUTH=PLAIN AUTH=LOGIN] Dovecot ready.
\end{lstlisting}

\begin{lstlisting}[language=tex, breaklines, frame=single, caption={IMAP settings}, label=lst:protcol2, float, floatplacement=h]
OK [CAPABILITY IMAP4rev1 LITERAL+ SASL-IR LOGIN-REFERRALS ID ENABLE IDLE SORT SORT=DISPLAY THREAD=REFERENCES THREAD=REFS THREAD=ORDEREDSUBJECT MULTIAPPEND URL-PARTIAL CATENATE UNSELECT CHILDREN NAMESPACE UIDPLUS LIST-EXTENDED I18NLEVEL=1 CONDSTORE QRESYNC ESEARCH ESORT SEARCHRES WITHIN CONTEXT=SEARCH LIST-STATUS BINARY MOVE SPECIAL-USE] Logged in
\end{lstlisting}

Listing \ref{lst:authenticateRequest} shows the request for authentication by the server.
The client answers with a Base64 encoded string that is shown in listing \ref{lst:authenticatePayload}.

%The authentication protocol exchange consists of a series of
%server challenges and client responses that are specific to the
%authentication mechanism.  A server challenge consists of a
%command continuation request response with the "+" token followed
%by a BASE64 encoded string.  The client response consists of a
%single line consisting of a BASE64 encoded string.  If the client
%wishes to cancel an authentication exchange, it issues a line
%consisting of a single "*".  If the server receives such a
%response, it MUST reject the AUTHENTICATE command by sending a
%tagged BAD response.

\begin{lstlisting}[language=tex, breaklines, frame=single, caption={Request for authentication by server}, label=lst:authenticateRequest, float, floatplacement=h]
authenticate PLAIN
\end{lstlisting}

\begin{lstlisting}[language=tex, breaklines, frame=single, caption={Base64 encoded message}, label=lst:authenticatePayload, float, floatplacement=h]
GNuXzA4QGV4MS5jbjJsYWIuY24udHV3aWVuLmFjLmF0AFhpdm9qaWx1cGE4
\end{lstlisting}

\subsubsection{Decoding the message}
The message can be decoded with any Base64 decoder tool, the result can be seen in Listing \ref{lst:authenticatePayloadDecoded}.
For this exercise, an \href{https://www.base64decode.org/}{online tool} (\url{https://www.base64decode.org/}) was used.
The client answers to the challenge of the server with its username and password, both transmitted as plain-text.
\begin{lstlisting}[language=tex, breaklines, frame=single, caption={Base64 decoded message}, label=lst:authenticatePayloadDecoded, float, floatplacement=h]
cn_08@ex1.cn2lab.cn.tuwien.ac.atXivojilupa8
\end{lstlisting}

\subsection{Decoded Password}
The password can be deducted by removal of the mail address from the Base64 decoded string and is \texttt{Xivojilupa8}.

\subsection{Strategies to secure the communication}
\subsubsection{Transmission over TLS}
If a plain-text authentication process is obligatory, one can resort to transmission of the plain-text password over a channel that is protected by TLS (Transport Layer Security). 
This protects from passive eavesdropping, but doesn't prevent man-in-the-middle attacks.

\subsubsection{Salted Challenge Response Authentication Mechanism (SCRAM)}
Preventing man-in-the-middle attacks presumes that the endpoints assures their identities to each other.
Therefore, SCRAM (see RFC5802) in combination with certificates and TLS can be used as an alternative to the PLAIN authentication mechanism.
Furthermore, it protects against offline attacks on the password and user database by storing just password hashes. 

%\begin{lstlisting}[language=tex, frame=single, caption={Code for adding a picture}, label=lst:figure, float, floatplacement=h]
%\begin{figure}[ht]
% \centering
% \includegraphics[height=6cm]{images/loremipsum.png}
% \caption{Descriptive text.} 
% \label{fig:lorem}
%\end{figure} 
%\end{lstlisting}
%
%Hint: if you add references to items,
%then \LaTeX needs to be executed two times. The first run is for writing down every reference and the second run for actually outputting the correct reference instead of \verb+\cref{XX}+.
%If your document contains ?? instead of a reference, you forgot the second compile run.
%
%\begin{figure}[ht]
% \centering % centering figure 
% \includegraphics[height=6cm]{images/loremipsum.png} % importing figure
% \caption{Don't forget to find a fitting caption for your graphics.} 
% \label{fig:lorem} % labeling to refer it inside the text
%\end{figure} 

%----------------------------------------------------------------------------------------
%	SECTION X
%---------------------------------------------------------------------------------------


%%%%%%%%%%%%%%%%%%%%%%%%%%%%%%%%%%%%%%%%%%%%%%%
\end{document}
