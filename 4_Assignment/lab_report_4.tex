%%%%%%%%%%%%%%%%%%%%%%%%%%%%%%%%%%%%%%%%%
% CN2 Labreport template
%
% License:
% CC BY-NC-SA 3.0 (http://creativecommons.org/licenses/by-nc-sa/3.0/)
%
%%%%%%%%%%%%%%%%%%%%%%%%%%%%%%%%%%%%%%%%%

\documentclass[parskip=full]{scrartcl}

\usepackage{siunitx}  % Provides the \SI{}{} command for typesetting SI units
\usepackage{graphicx} % Required for the inclusion of images
\usepackage{booktabs} % nicer tables
\usepackage[noabbrev]{cleveref} % automatic references
\usepackage{listings} % typeset code

\crefname{lstlisting}{listing}{listings} % for referencing code
\Crefname{lstlisting}{Listing}{Listings} % for referencing code

\usepackage[headsepline]{scrlayer-scrpage} % header
\ohead{Group 08} % right part of header
\ihead{Assignment 4} % left part of header

\lstset{basicstyle=\ttfamily} % monospaced font in listing

\usepackage{lstautogobble}  % Fix relative indenting
\usepackage{color}          % Code coloring
\usepackage{zi4}            % Nice font
\usepackage{enumitem}
\usepackage{titlesec}

\definecolor{bluekeywords}{rgb}{0.13, 0.13, 1}
\definecolor{greencomments}{rgb}{0, 0.5, 0}
\definecolor{redstrings}{rgb}{0.9, 0, 0}
\definecolor{graynumbers}{rgb}{0.5, 0.5, 0.5}
\titlespacing\section{0pt}{12pt plus 4pt minus 2pt}{0pt plus 2pt minus 2pt}
\titlespacing\subsection{0pt}{12pt plus 4pt minus 2pt}{0pt plus 2pt minus 2pt}
\titlespacing\subsubsection{0pt}{12pt plus 4pt minus 2pt}{0pt plus 2pt minus 2pt}
\usepackage{listings}
\lstset{
    autogobble,
    columns=fullflexible,
    showspaces=false,
    showtabs=false,
    breaklines=true,
    showstringspaces=false,
    breakatwhitespace=true,
    escapeinside={(*@}{@*)},
    commentstyle=\color{greencomments},
    keywordstyle=\color{bluekeywords},
    stringstyle=\color{redstrings},
    numberstyle=\color{graynumbers},
    basicstyle=\ttfamily\footnotesize,
    frame=l,
    framesep=12pt,
    xleftmargin=12pt,
    tabsize=4,
    captionpos=b
}


%----------------------------------------------------------------------------------------
%	DOCUMENT INFORMATION
%----------------------------------------------------------------------------------------

\begin{document}
\begin{titlepage}
    \centering
    \vspace*{2cm}
    {\Huge \textbf{Communication Networks 2}}\\
    SS 2019\\
    \vspace*{1cm}
    {\Large Assignment 4}
    \\\vspace*{3cm}
    {\Large \textbf{Group 08}}\\
    \vspace*{1cm}
    {\large 
        \begin{tabular}{l c c}
            Name & Mat.Nummer \\ \hline
            Constantin SCHIEBER & 01228774 \\
            Andreas HIRTENLEHNER & 01327273
        \end{tabular}
    }
    \\\vspace*{7cm}
    \today
\end{titlepage}

%----------------------------------------------------------------------------------------
%	SECTION 1
%----------------------------------------------------------------------------------------

\section{Description of the Solution}

We implemented a ns3 simulation based on the given specification and with the help of the examples provided in the ns3 documentation.
Furthermore, the generation of an XML-File for the tool \texttt{NetAnim} is implemented to visualize the created network.
We create another \texttt{NetAnim} compatible XML-File that contains the routing tables of the network nodes.
Static routing tables are computed at compile time with the \texttt{Ipv4GlobalRoutingHelper} library.

The \texttt{EnablePcapAll} function enables pcap output on all devices in the network. 
Note that this function needs to be invoked on \texttt{p2p} and \texttt{csma} device types separately.
This generates files of the type \texttt{prefix-A-B} where \texttt{A} is the node-id and \texttt{B} is the device-id. 
We expect 3 nodes with ID 0 to 2 and two devices with IDs 1 and 2. 
A third device ID 0 is present in the routing tables and refers to the local loopback interface. 
Subsequently, four pcap files are generated:

\begin{itemize}[noitemsep]
    \item \texttt{assignment4-0-1}
    \item \texttt{assignment4-1-1}
    \item \texttt{assignment4-1-2}
    \item \texttt{assignment4-2-1}
\end{itemize}

\subsection{Expected Results}
We expect about 9 seconds of recorded data. 
\subsubsection{Node 0, Interface 1}
Observations at Node 0, Interface 1 should show an outgoing \texttt{ICMP} packet to the address \texttt{10.0.148.2} every 0.2 seconds with a length of 1052 bytes (1024 bytes payload + 28 bytes protocol overhead).
An incoming \texttt{ICMP} packet from the source \texttt{10.0.148.2} to the destination \texttt{10.0.20.1} should arrive after a time of approximately $30ms + 2 \cdot 2ms = 64ms$.
The \texttt{ttl} of the incoming packet should be decreased by one, as Node 2 is traversed.
The sent packets should use P2P Frames. 

\subsubsection{Node 1, Interface 1 (P2P)}
Interface 1 on Node 1 receives messages from Node 0 every 0.2 seconds.
These messages are then forwarded into the subnet \texttt{10.0.148.0} and the answer from the pinged host should (this time relayed from Interface 2) be received after 4ms.
Packets should still use P2P Frames.
\subsubsection{Node 1, Interface 2 (CSMA)}
Interface 2 on Node 1 forwards the packet to its destination \texttt{10.0.148.2} and should record an answer from the destination after 4ms.
Packets should have been switched to Ethernet Frames, which most likely will increase the total length of the packets.
\subsubsection{Node 2, Interface 1 (CSMA)}
Node 2 should record a packet on Interface 1 every 0.2 seconds and an outgoing answer to this packet should be recorded immediately.
Packets should still use Ethernet Frames.

\subsubsection{Output of Ping}
We expect a constant output from the ping tool, i.e. 1032 transmitted bytes, a ttl of 63 and a time of 64ms.

\subsection{Observed Results and Explanation of Deviation}
The observed results match our expectation for the most part. 
We didn't consider that Nodes 1 and 2 may use the Address Resolution Protocol (ARP) at the start of the simulation.
The total packet length under the usage of P2P Frames is \textbf{1054} and not 1052 (\texttt{20 Byte Ipv4 + 8 Byte ICMP + 2 Byte P2P}) as the size of the P2P frame was not considered.
The total packet length for Ethernet Frames increases to \textbf{1070} (\texttt{18 Byte Ethernet + 20 Byte IPv4 + 8 Byte ICMP}).

The ping tool reports a time of 83ms for the first ping (Listing \ref{lst:ns3Ping}), this can be attributed to the fact that ARP packets are exchanged in the beginning and the first ping is delayed by that (Listing \ref{lst:ns3ARP}). 
\begin{lstlisting}[language=tex, breaklines, frame=single, caption={Ping output from the ns3 simulation}, label=lst:ns3Ping, float, floatplacement=h]
PING  10.0.148.2 56(84) bytes of data.
1032 bytes from 10.0.148.2: icmp_seq=0 ttl=63 time=83 ms
1032 bytes from 10.0.148.2: icmp_seq=1 ttl=63 time=65 ms
...
\end{lstlisting}

\begin{lstlisting}[language=tex, breaklines, frame=single, caption={ARP messages that delay the first ping}, label=lst:ns3ARP, float, floatplacement=h]
1	0.000000	00:00:00_00:00:03	Broadcast	ARP	64		64					
2	0.004012	00:00:00_00:00:04	00:00:00_00:00:03	ARP	64		64									
4	0.017114	00:00:00_00:00:04	Broadcast	ARP	64		64									
5	0.017114	00:00:00_00:00:03	00:00:00_00:00:04	ARP	64		64									

\end{lstlisting}

%%%%%%%%%%%%%%%%%%%%%%%%%%%%%%%%%%%%%%%%%%%%%%%
\end{document}
