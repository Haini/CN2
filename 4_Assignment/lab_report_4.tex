%%%%%%%%%%%%%%%%%%%%%%%%%%%%%%%%%%%%%%%%%
% CN2 Labreport template
%
% License:
% CC BY-NC-SA 3.0 (http://creativecommons.org/licenses/by-nc-sa/3.0/)
%
%%%%%%%%%%%%%%%%%%%%%%%%%%%%%%%%%%%%%%%%%

\documentclass[parskip=full]{scrartcl}

\usepackage{siunitx}  % Provides the \SI{}{} command for typesetting SI units
\usepackage{graphicx} % Required for the inclusion of images
\usepackage{booktabs} % nicer tables
\usepackage[noabbrev]{cleveref} % automatic references
\usepackage{listings} % typeset code

\crefname{lstlisting}{listing}{listings} % for referencing code
\Crefname{lstlisting}{Listing}{Listings} % for referencing code

\usepackage[headsepline]{scrlayer-scrpage} % header
\ohead{Group 08} % right part of header
\ihead{Assignment 4} % left part of header

\lstset{basicstyle=\ttfamily} % monospaced font in listing

\usepackage{lstautogobble}  % Fix relative indenting
\usepackage{color}          % Code coloring
\usepackage{zi4}            % Nice font
\usepackage{enumitem}
\usepackage{titlesec}

\definecolor{bluekeywords}{rgb}{0.13, 0.13, 1}
\definecolor{greencomments}{rgb}{0, 0.5, 0}
\definecolor{redstrings}{rgb}{0.9, 0, 0}
\definecolor{graynumbers}{rgb}{0.5, 0.5, 0.5}
\titlespacing\section{0pt}{12pt plus 4pt minus 2pt}{0pt plus 2pt minus 2pt}
\titlespacing\subsection{0pt}{12pt plus 4pt minus 2pt}{0pt plus 2pt minus 2pt}
\titlespacing\subsubsection{0pt}{12pt plus 4pt minus 2pt}{0pt plus 2pt minus 2pt}
\usepackage{listings}
\lstset{
    autogobble,
    columns=fullflexible,
    showspaces=false,
    showtabs=false,
    breaklines=true,
    showstringspaces=false,
    breakatwhitespace=true,
    escapeinside={(*@}{@*)},
    commentstyle=\color{greencomments},
    keywordstyle=\color{bluekeywords},
    stringstyle=\color{redstrings},
    numberstyle=\color{graynumbers},
    basicstyle=\ttfamily\footnotesize,
    frame=l,
    framesep=12pt,
    xleftmargin=12pt,
    tabsize=4,
    captionpos=b
}


%----------------------------------------------------------------------------------------
%	DOCUMENT INFORMATION
%----------------------------------------------------------------------------------------

\begin{document}
\begin{titlepage}
    \centering
    \vspace*{2cm}
    {\Huge \textbf{Communication Networks 2}}\\
    SS 2019\\
    \vspace*{1cm}
    {\Large Assignment 4}
    \\\vspace*{3cm}
    {\Large \textbf{Group 08}}\\
    \vspace*{1cm}
    {\large 
        \begin{tabular}{l c c}
            Name & Mat.Nummer \\ \hline
            Constantin SCHIEBER & 01228774 \\
            Andreas HIRTENLEHNER & 01327273
        \end{tabular}
    }
    \\\vspace*{7cm}
    \today
\end{titlepage}

%----------------------------------------------------------------------------------------
%	SECTION 1
%----------------------------------------------------------------------------------------

\section{Description of the Solution}
We implement a ns3 simulation based on the given specification.
This was done based on the examples provided in the ns3 documentation.
Furthermore, the generation of an XML-File for the tool \texttt{NetAnim} is implemented to visualize the created network.
We create another \texttt{NetAnim} compatible XML-File that contains the routing tables of the network nodes.
Static routing tables are computed at compile time with the \texttt{Ipv4GlobalRoutingHelper} library.

The \texttt{EnablePcapAll} function enables pcap output on all devices in the network. 
Note that this function needs to be invoked on \texttt{p2p} and \texttt{csma} device types separately.
This generates files of the type \texttt{prefix-A-B} where \texttt{A} is the node-id and \texttt{B} is the device-id. 
We expect 3 nodes with ID 0 to 2 and two devices IDs 1 and 2. 
A third device ID 0 is present in the routing tables and refers to the local loop back interface. 
Subsequently, four pcap files are generated during simulation:

\begin{itemize}[noitemsep]
    \item \texttt{assignment4-0-1}
    \item \texttt{assignment4-1-1}
    \item \texttt{assignment4-1-2}
    \item \texttt{assignment4-2-1}
\end{itemize}

\subsubsection{Creation of Nodes}
Three nodes are created and assigned to two different node containers.
The node that needs to implement \texttt{csma} and \texttt{p2p} functionality is assigned to both node containers. 
An overview over the procedure is given in Listing \ref{lst:ns3Nodes}.
\begin{lstlisting}[language=c++, breaklines, frame=single, caption={Setup of the nodes and connections}, label=lst:ns3Nodes, float, floatplacement=h]
  //Create node container with 3 nodes
  NodeContainer c;
  c.Create (3);
  ...
  //Assign nodes so that IPs get assigned in the correct order later on
  NodeContainer n0n1 = NodeContainer (c.Get (1), c.Get (0));
  NodeContainer n1n2 = NodeContainer (c.Get (1), c.Get (2));
  ...
  //Configure settings for P2P connection
  PointToPointHelper p2p;
  p2p.SetDeviceAttribute ("DataRate", StringValue ("13Mbps"));
  p2p.SetChannelAttribute ("Delay", StringValue ("30ms"));
  NetDeviceContainer d0d1 = p2p.Install (n0n1);
  ...
  //Configure settings for CSMA connection 
  CsmaHelper csma;
  csma.SetChannelAttribute ("DataRate", StringValue ("90Mbps"));
  csma.SetChannelAttribute ("Delay", StringValue ("2ms"));
  NetDeviceContainer d1d2 = csma.Install (n1n2);
\end{lstlisting}
\subsubsection{Creation of Application}
After the network for the simulation is established, a ping application needs to be implemented on Node 1 (Listing \ref{lst:ns3PingSetup}).
 
\begin{lstlisting}[language=c++, breaklines, frame=single, caption={Configuration and installation of ping application in Node 1}, label=lst:ns3PingSetup, float, floatplacement=h]
  ...
  //Setup the ping application
  uint32_t packetSize = 1024;
  Time interPacketInterval = Seconds (0.2);
  V4PingHelper ping ("10.0.148.2");

  /* Start and Stop get overwritten in apps.start / apps.stop anyways */
  ping.SetAttribute("StartTime", TimeValue (Seconds (1)));
  ping.SetAttribute("StopTime", TimeValue (Seconds (10)));
  /* Interval of 0.2 seconds */
  ping.SetAttribute ("Interval", TimeValue (interPacketInterval));
  /* Packet Size of 1024 + 28 byte overhead = 1052 bytes total */
  ping.SetAttribute ("Size", UintegerValue (packetSize));
  ...
  /* Configures who is pinging and in which time frame */
  ApplicationContainer apps = ping.Install (c.Get (0));
  apps.Start (Seconds (1.0));
  apps.Stop (Seconds (10.0));
  ...
\end{lstlisting}


\subsection{Expected Results}
We expect about 9 seconds of recorded data. 
\subsubsection{Node 0, Interface 1}
Observations at Node 0, Interface 1 should show an outgoing \texttt{ICMP} packet to the address \texttt{10.0.148.2} every 0.2 seconds with a length of 1052 bytes (1024 bytes payload + 28 bytes protocol overhead).
An incoming \texttt{ICMP} packet from the source \texttt{10.0.148.2} to the destination \texttt{10.0.20.1} should arrive after a time of approximately \texttt{(30ms + 2ms * 2) = 64ms}.
The \texttt{ttl} of the incoming packet should be decreased by one, as Node 2 is traversed.
The sent packets should be using P2P Frames. 

\subsubsection{Node 1, Interface 1 (P2P)}
Interface 1 on Node 1 receives messages from Node 0 every 0.2 seconds.
These messages are then forwarded into the subnet \texttt{10.0.148.0} and the answer from the pinged host should (this time relayed from Interface 2) be received after 4ms.
Packets should still be using P2P Frames.
\subsubsection{Node 1, Interface 2 (CSMA)}
Interface 2 on Node 1 forwards the packet to its destination, \texttt{10.0.148.2} and should record an answer from the destination after 4ms.
Packets should have been switched to Ethernet Frames, which most likely will increase the total length of the packets.
\subsubsection{Node 2, Interface 1 (CSMA)}
Node 2 should record a packet on Interface 1 every 0.2 seconds and an outgoing answer to this packet should be recorded immediately.
Packets should be still using Ethernet Frames.

\subsubsection{Output of Ping}
We expect a constant output from the ping tool, i.e. 1032 transmitted bytes, a ttl of 63 and a time of 64ms.

\subsection{Observed Results and Explanation of Deviation}
The simulation results match our expectation for the most part. 
In the following, we discuss the observed deviations.

\subsubsection{Total Packet Length of Frames}
The total packet length under the usage of P2P Frames is \textbf{1054} and not 1052 (\texttt{20 Byte Ipv4 + 8 Byte ICMP + 2 Byte P2P}) as the size of the P2P frame was not considered.

The total packet length for Ethernet Frames increases to \textbf{1070} (\texttt{18 Byte Ethernet + 20 Byte IPv4 + 8 Byte ICMP}).

\subsubsection{Unexpected Address Resolution Protocol (ARP) Messages}
We didn't consider that Nodes 1 and 2 may use ARP at the start of the simulation.
The ping tool reports a time of 83ms for the first ping (Listing \ref{lst:ns3Ping}), this can be attributed to the fact that ARP packets are exchanged in the beginning and the first ping is delayed by that (Listing \ref{lst:ns3ARP}). 
\begin{lstlisting}[language=tex, breaklines, frame=single, caption={Ping output from the ns3 simulation}, label=lst:ns3Ping, float, floatplacement=h]
PING  10.0.148.2 56(84) bytes of data.
1032 bytes from 10.0.148.2: icmp_seq=0 ttl=63 time=83 ms
1032 bytes from 10.0.148.2: icmp_seq=1 ttl=63 time=65 ms
...
\end{lstlisting}

\begin{lstlisting}[language=tex, breaklines, frame=single, caption={ARP messages that delay the first ping}, label=lst:ns3ARP, float, floatplacement=h]
1	0.000000	00:00:00_00:00:03	Broadcast	ARP	64		64					
2	0.004012	00:00:00_00:00:04	00:00:00_00:00:03	ARP	64		64									
4	0.017114	00:00:00_00:00:04	Broadcast	ARP	64		64									
5	0.017114	00:00:00_00:00:03	00:00:00_00:00:04	ARP	64		64									
\end{lstlisting}
\subsubsection{Round Trip Time (RTT)}
We also observe a longer than expected RTT of the ping (65ms) / timestamps of received packets in Wireshark (65,488ms).
This can be attributed to the Transmission Delay: $D_T = \frac{N}{R}$.
$N$ is the number of transmitted bits, $R$ is the available data rate.
It refers to the delay that is introduced to push all bits into the wire.

For our setup the $D_T + delay$ should be $2*(\frac{1054b*8}{13Mbps} + \frac{1070b*8}{90Mbps})+64ms \approx 65,487ms$.
This value comes pretty close to the measured $65,488ms$ in Wireshark and explains the deviation from our expectation.

\subsubsection{Faulty Checksums}
Wireshark reports that the calculated checksums and frame check sequences are incorrect. 
This does not have direct consequences on the functionality of the simulation, albeit Wireshark can't match the response frames for ICMP and will highlight the wrong checksum values.
Calculation of checksums needs to be explicitly enabled in the ns3 program by setting the \texttt{ChecksumEnabled} value to true (Listing \ref{lst:ns3Checksum})

\begin{lstlisting}[language=c++, breaklines, frame=single, caption={Enable checksum calculation}, label=lst:ns3Checksum, float, floatplacement=h]
GlobalValue::Bind ("ChecksumEnabled", BooleanValue (true));
\end{lstlisting}

%%%%%%%%%%%%%%%%%%%%%%%%%%%%%%%%%%%%%%%%%%%%%%%
\end{document}
