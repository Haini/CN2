%%%%%%%%%%%%%%%%%%%%%%%%%%%%%%%%%%%%%%%%%
% CN2 Labreport template
%
% License:
% CC BY-NC-SA 3.0 (http://creativecommons.org/licenses/by-nc-sa/3.0/)
%
%%%%%%%%%%%%%%%%%%%%%%%%%%%%%%%%%%%%%%%%%

\documentclass[parskip=full]{scrartcl}

\usepackage{siunitx}  % Provides the \SI{}{} command for typesetting SI units
\usepackage{graphicx} % Required for the inclusion of images
\usepackage{booktabs} % nicer tables
\usepackage[noabbrev]{cleveref} % automatic references
\usepackage{listings} % typeset code

\crefname{lstlisting}{listing}{listings} % for referencing code
\Crefname{lstlisting}{Listing}{Listings} % for referencing code

\usepackage[headsepline]{scrlayer-scrpage} % header
\ohead{Group 08} % right part of header
\ihead{Assignment 2} % left part of header

\lstset{basicstyle=\ttfamily} % monospaced font in listing

\usepackage{lstautogobble}  % Fix relative indenting
\usepackage{color}          % Code coloring
\usepackage{zi4}            % Nice font

\definecolor{bluekeywords}{rgb}{0.13, 0.13, 1}
\definecolor{greencomments}{rgb}{0, 0.5, 0}
\definecolor{redstrings}{rgb}{0.9, 0, 0}
\definecolor{graynumbers}{rgb}{0.5, 0.5, 0.5}

\usepackage{listings}
\lstset{
    autogobble,
    columns=fullflexible,
    showspaces=false,
    showtabs=false,
    breaklines=true,
    showstringspaces=false,
    breakatwhitespace=true,
    escapeinside={(*@}{@*)},
    commentstyle=\color{greencomments},
    keywordstyle=\color{bluekeywords},
    stringstyle=\color{redstrings},
    numberstyle=\color{graynumbers},
    basicstyle=\ttfamily\footnotesize,
    frame=l,
    framesep=12pt,
    xleftmargin=12pt,
    tabsize=4,
    captionpos=b
}


%----------------------------------------------------------------------------------------
%	DOCUMENT INFORMATION
%----------------------------------------------------------------------------------------

\begin{document}
\begin{titlepage}
    \centering
    \vspace*{2cm}
    {\Huge \textbf{Communication Networks 2}}\\
    SS 2017\\
    \vspace*{1cm}
    {\Large Assignment 4}
    \\\vspace*{3cm}
    {\Large \textbf{Group 08}}\\
    \vspace*{1cm}
    {\large 
        \begin{tabular}{l c c}
            Name & Mat.Nummer \\ \hline
            Constantin SCHIEBER & 01228774 \\
            Andreas HIRTENLEHNER & 01327273
        \end{tabular}
    }
    \\\vspace*{7cm}
    \today
\end{titlepage}

%----------------------------------------------------------------------------------------
%	SECTION 1
%----------------------------------------------------------------------------------------

\section{Description of the Solution}
We implement a ns3 simulation based on the given specification and with the help of the examples provided in the ns3 documentation.
Furthermore, the generation of an XML-File for the tool \texttt{NetAnim} is implemented to visualize the created network.
We create another \texttt{NetAnim} compatible XML-File that contains the routing tables of the network nodes.
Static routing tables are computed at compile time with the \texttt{Ipv4GlobalRoutingHelper} library.

\begin{lstlisting}[language=c, breaklines, frame=single, caption={Setup of the Nodes}, label=lst:setupNodes, float, floatplacement=h]
\end{lstlisting}

\subsection{Expected Results}
\subsubsection{Observations at Node 1}
Observations at Node 1 should show an outgoing \texttt{ICMP} packet to the address \texttt{10.0.148.2} every 0.2 seconds with a length of 1054 bytes (1024 bytes payload + 28 bytes protocol overhead).
An incoming \texttt{ICMP} packet from the source \texttt{10.0.148.2} to the destination \texttt{10.0.20.1} should arrive after a time of approximately \texttt{(30ms + 2ms * 2) = 64ms}.
The \texttt{ttl} of the incoming packet should be decreased by one, as Node 2 is traversed. 
\subsubsection{Observations at Node 2}
Node 2 should observe a packet from \texttt{10.0.20.1} every \texttt{0.2s*t + 30ms = t*2.3s} and an answer to that packet form \texttt{10.0.148.2} every \texttt{0.2s*t + 30ms + 2*2ms = t*2.34s} 
\subsubsection{Observations at Node 3}

\subsection{Observed Results}
\subsection{Explanation of Deviation}

%%%%%%%%%%%%%%%%%%%%%%%%%%%%%%%%%%%%%%%%%%%%%%%
\end{document}
